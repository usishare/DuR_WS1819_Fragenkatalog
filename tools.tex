% -----
% achim
% -----
%
%
%
\newenvironment{achim}
{%
  \begingroup
    \par
    \color{Blue}%
}%
{%
  \endgroup
}%

% -------
% karsten
% -------
%
%
%
\newenvironment{karsten}
{%
  \begingroup
    \par
    \color{Red}%
}%
{%
  \endgroup
}%

% ------
% moritz
% ------
%
%
%
\newenvironment{moritz}
{%
  \begingroup
    \par
    \color{Green}%
}%
{%
  \endgroup
}%

% ----
% KV16
% ----
%
% This environment draws a 4x4 KV-map within a tikz environment.
%
\newenvironment{KV16}
{%
  % ----------------------------------------------------------------------------
  % Public functions                                            Public functions
  % ----------------------------------------------------------------------------
  % --------
  % nolabels
  % --------
  %
  % disable input labels
  %
  \newcommand{\nolabels}
  {%
    \renewcommand{\showlabels}{FALSE}%
  }%
  % ---------
  % noindices
  % ---------
  %
  % disable index numbers
  %
  \newcommand{\noindices}
  {%
    \renewcommand{\showindices}{FALSE}%
  }%
  % -----------
  % clearoutput
  % -----------
  %
  % empty all function values
  %
  \newcommand{\clearoutput}
  {%
    \setoutput{0}{}%
    \setoutput{1}{}%
    \setoutput{2}{}%
    \setoutput{3}{}%
    \setoutput{4}{}%
    \setoutput{5}{}%
    \setoutput{6}{}%
    \setoutput{7}{}%
    \setoutput{8}{}%
    \setoutput{9}{}%
    \setoutput{10}{}%
    \setoutput{11}{}%
    \setoutput{12}{}%
    \setoutput{13}{}%
    \setoutput{14}{}%
    \setoutput{15}{}%
  }%
  % --------
  % function
  % --------
  %
  % #1  name of the function
  %
  \newcommand{\function}[1]
  {%
    \renewcommand{\valuefname}{##1}%
  }%
  % ------
  % inputA
  % ------
  %
  % set the name of the input variable that toggles with the lowest frequency
  %
  % #1  name of the variable
  %
  \newcommand{\inputA}[1]
  {%
    \renewcommand{\valueinputA}{##1}%
  }%
  % ------
  % inputB
  % ------
  %
  % #1  name of the variable
  %
  \newcommand{\inputB}[1]
  {%
    \renewcommand{\valueinputB}{##1}%
  }%
  % ------
  % inputC
  % ------
  %
  % #1  name of the variable
  %
  \newcommand{\inputC}[1]
  {%
    \renewcommand{\valueinputC}{##1}%
  }%
  % ------
  % inputD
  % ------
  %
  % set the name of the input variable that toggles with the highest frequency
  %
  % #1  name of the variable
  %
  \newcommand{\inputD}[1]
  {%
    \renewcommand{\valueinputD}{##1}%
  }%
  % -------
  % minterm
  % -------
  %
  % #1  minterm number 0 ... 15
  %
  \newcommand{\minterm}[1]
  {%
    \setoutput{##1}{1}%
  }%
  % -------
  % maxterm
  % -------
  %
  % #1  maxterm number 0 ... 15
  %
  \newcommand{\maxterm}[1]
  {%
    \setoutput{##1}{0}%
  }%
  % --------
  % dontcare
  % --------
  %
  % #1  don't care number 0 ... 15
  %
  \newcommand{\dontcare}[1]
  {%
    \setoutput{##1}{X}%
  }%
  % -----
  % areas
  % -----
  %
  % define areas to fill
  %
  \newcommand{\areas}[1]
  {%
    \renewcommand{\printareas}{##1}
  }%
  % --------
  % colorize
  % --------
  %
  % #1 top row          (1..m)
  % #2 left column      (1..n)
  % #3 width            (number of columns)
  % #4 height           (number of rows)
  % #5 top    reduction (pt, cm, ...)
  % #6 bottom reduction (pt, cm, ...)
  % #7 left   reduction (pt, cm, ...)
  % #8 right  reduction (pt, cm, ...)
  % #9 color
  %
  \newcommand{\colorize}[9]
  {%
    \begin{scope}[xshift=0cm + (##2 - 1) * 1cm, yshift=4cm - (##1 - 1) * 1cm]
      % background
      \ifthenelse{\equal{\showtype}{FILL}}%
      {%
        \fill
        [
          opacity=0.5,
          fill=##9,
          rounded corners
        ] (0cm + ##7, 0cm - ##5) rectangle (1cm * ##3 - ##8, -1cm * ##4 + ##6);
      }%
      {\relax}%
      % border
      \ifthenelse{\equal{\showtype}{DRAW}}%
      {%
        \draw
        [
          draw=##9,
          line width=1pt,
          rounded corners
        ] (0cm + ##7, 0cm - ##5) rectangle (1cm * ##3 - ##8, -1cm * ##4 + ##6);
      }%
      {\relax}%
    \end{scope}
  }%
  % ----------------------------------------------------------------------------
  % Internal functions                                        Internal functions
  % ----------------------------------------------------------------------------
  % ---------
  % setoutput
  % ---------
  %
  % used my minterm, maxterm and dontcare
  %
  % #1  index: 0 .. 15
  % #2  text
  %
  \newcommand{\setoutput}[2]
  {%
    % map index to matrix position
    \ifthenelse{\equal{##1}{0}} {\renewcommand{\valueoutputAA}{##2}}{\relax}%
    \ifthenelse{\equal{##1}{1}} {\renewcommand{\valueoutputAB}{##2}}{\relax}%
    \ifthenelse{\equal{##1}{2}} {\renewcommand{\valueoutputAD}{##2}}{\relax}%
    \ifthenelse{\equal{##1}{3}} {\renewcommand{\valueoutputAC}{##2}}{\relax}%
    \ifthenelse{\equal{##1}{4}} {\renewcommand{\valueoutputBA}{##2}}{\relax}%
    \ifthenelse{\equal{##1}{5}} {\renewcommand{\valueoutputBB}{##2}}{\relax}%
    \ifthenelse{\equal{##1}{6}} {\renewcommand{\valueoutputBD}{##2}}{\relax}%
    \ifthenelse{\equal{##1}{7}} {\renewcommand{\valueoutputBC}{##2}}{\relax}%
    \ifthenelse{\equal{##1}{8}} {\renewcommand{\valueoutputDA}{##2}}{\relax}%
    \ifthenelse{\equal{##1}{9}} {\renewcommand{\valueoutputDB}{##2}}{\relax}%
    \ifthenelse{\equal{##1}{10}}{\renewcommand{\valueoutputDD}{##2}}{\relax}%
    \ifthenelse{\equal{##1}{11}}{\renewcommand{\valueoutputDC}{##2}}{\relax}%
    \ifthenelse{\equal{##1}{12}}{\renewcommand{\valueoutputCA}{##2}}{\relax}%
    \ifthenelse{\equal{##1}{13}}{\renewcommand{\valueoutputCB}{##2}}{\relax}%
    \ifthenelse{\equal{##1}{14}}{\renewcommand{\valueoutputCD}{##2}}{\relax}%
    \ifthenelse{\equal{##1}{15}}{\renewcommand{\valueoutputCC}{##2}}{\relax}%
  }%
  % -----------
  % printlabels
  % -----------
  %
  % print all input labels
  %
  \newcommand{\printlabels}
  {%
    \draw[|-|] ( 4.3,  0.0) -- node[right] {\valueinputA} ( 4.3,  2.0);
    \draw[|-|] (-0.3,  1.0) -- node[left]  {\valueinputB} (-0.3,  3.0);
    \draw[|-|] ( 1.0, -0.3) -- node[below] {\valueinputD} ( 3.0, -0.3);
    \draw[|-|] ( 2.0,  4.3) -- node[above] {\valueinputC} ( 4.0,  4.3);
  }%
  % ------------
  % printindices
  % ------------
  %
  % print all index numbers
  %
  \newcommand{\printindices}
  {%
    \begin{scope}[xshift=2mm, yshift=8mm]
      % 1. row (top)
      \node at (0, 3) {{\tiny\sffamily  0}};
      \node at (1, 3) {{\tiny\sffamily  1}};
      \node at (2, 3) {{\tiny\sffamily  3}};
      \node at (3, 3) {{\tiny\sffamily  2}};
      % 2. row
      \node at (0, 2) {{\tiny\sffamily  4}};
      \node at (1, 2) {{\tiny\sffamily  5}};
      \node at (2, 2) {{\tiny\sffamily  7}};
      \node at (3, 2) {{\tiny\sffamily  6}};
      % 3. row
      \node at (0, 1) {{\tiny\sffamily 12}};
      \node at (1, 1) {{\tiny\sffamily 13}};
      \node at (2, 1) {{\tiny\sffamily 15}};
      \node at (3, 1) {{\tiny\sffamily 14}};
      % 4. row (bottom)
      \node at (0, 0) {{\tiny\sffamily  8}};
      \node at (1, 0) {{\tiny\sffamily  9}};
      \node at (2, 0) {{\tiny\sffamily 11}};
      \node at (3, 0) {{\tiny\sffamily 10}};
    \end{scope}
  }%
  % -----------
  % printoutput
  % -----------
  %
  % show all output values
  %
  \newcommand{\printoutput}
  {%
    \begin{scope}[xshift=0.5cm, yshift=0.5cm]
      % 1. row (top)
      \node at (0, 3) {\valueoutputAA};
      \node at (1, 3) {\valueoutputAB};
      \node at (2, 3) {\valueoutputAC};
      \node at (3, 3) {\valueoutputAD};
      % 2. row
      \node at (0, 2) {\valueoutputBA};
      \node at (1, 2) {\valueoutputBB};
      \node at (2, 2) {\valueoutputBC};
      \node at (3, 2) {\valueoutputBD};
      % 3. row
      \node at (0, 1) {\valueoutputCA};
      \node at (1, 1) {\valueoutputCB};
      \node at (2, 1) {\valueoutputCC};
      \node at (3, 1) {\valueoutputCD};
      % 4. row (bottom)
      \node at (0, 0) {\valueoutputDA};
      \node at (1, 0) {\valueoutputDB};
      \node at (2, 0) {\valueoutputDC};
      \node at (3, 0) {\valueoutputDD};
    \end{scope}
  }%
  % ----------
  % printareas
  % ----------
  %
  % show all colorized areas
  %
  \newcommand{\printareas}{\relax}%
  % ----------------------------------------------------------------------------
  % Attributes                                                        Attributes
  % ----------------------------------------------------------------------------
  % some common colors
  \definecolor{C1}{named}{Red}%
  \definecolor{C2}{named}{Green}%
  \definecolor{C3}{named}{Blue}%
  \definecolor{C4}{named}{Sepia}%
  \definecolor{C5}{named}{Cyan}%
  \definecolor{C6}{named}{Magenta}%
  \definecolor{C7}{named}{Yellow}%
  \definecolor{C8}{named}{Gray}%
  % flags
  \newcommand{\showlabels}{TRUE}%
  \newcommand{\showindices}{TRUE}%
  % name of the function
  \newcommand{\valuefname}{\relax}%
  % names of the four input variables
  \newcommand{\valueinputA}{$e_3$}%
  \newcommand{\valueinputB}{$e_2$}%
  \newcommand{\valueinputC}{$e_1$}%
  \newcommand{\valueinputD}{$e_0$}%
  % output values (matrix positions)
  \newcommand{\valueoutputAA}{0}% 0
  \newcommand{\valueoutputAB}{0}% 1
  \newcommand{\valueoutputAC}{0}% 3
  \newcommand{\valueoutputAD}{0}% 2
  \newcommand{\valueoutputBA}{0}% 4
  \newcommand{\valueoutputBB}{0}% 5
  \newcommand{\valueoutputBC}{0}% 7
  \newcommand{\valueoutputBD}{0}% 6
  \newcommand{\valueoutputCA}{0}% 12
  \newcommand{\valueoutputCB}{0}% 13
  \newcommand{\valueoutputCC}{0}% 15
  \newcommand{\valueoutputCD}{0}% 14
  \newcommand{\valueoutputDA}{0}% 8
  \newcommand{\valueoutputDB}{0}% 9
  \newcommand{\valueoutputDC}{0}% 11
  \newcommand{\valueoutputDD}{0}% 10
}%
{%
  \begingroup
    % show colored areas first (lowest layer)
    \begin{scope}
      % clip colored areas
      \clip (-0.15, -0.15) rectangle (4.15, 4.15);
      % show background
      {\newcommand{\showtype}{FILL}\printareas}%
      % show border
      {\newcommand{\showtype}{DRAW}\printareas}%
    \end{scope}
    % show grid
    \draw (0, 0) grid (4, 4);
    % name of the function
    \ifthenelse{\equal{\valuefname}{\relax}}{\relax}{\node at (-0.5, 4.5) {\valuefname};}
    % labels
    \ifthenelse{\equal{\showlabels}{TRUE}}{\printlabels}{\relax}
    % indices
    \ifthenelse{\equal{\showindices}{TRUE}}{\printindices}{\relax}
    % output values
    \printoutput
  \endgroup
}%

% ----
% KV32
% ----
%
% This environment draws a 4x8 KV-map within a tikz environment.
%
\newenvironment{KV32}
{%
  % ----------------------------------------------------------------------------
  % Public functions                                            Public functions
  % ----------------------------------------------------------------------------
  % --------
  % nolabels
  % --------
  %
  % disable input labels
  %
  \newcommand{\nolabels}
  {%
    \renewcommand{\showlabels}{FALSE}%
  }%
  % ---------
  % noindices
  % ---------
  %
  % disable index numbers
  %
  \newcommand{\noindices}
  {%
    \renewcommand{\showindices}{FALSE}%
  }%
  % -----------
  % clearoutput
  % -----------
  %
  % empty all function values
  %
  \newcommand{\clearoutput}
  {%
    \setoutput{0}{}%
    \setoutput{1}{}%
    \setoutput{2}{}%
    \setoutput{3}{}%
    \setoutput{4}{}%
    \setoutput{5}{}%
    \setoutput{6}{}%
    \setoutput{7}{}%
    \setoutput{8}{}%
    \setoutput{9}{}%
    \setoutput{10}{}%
    \setoutput{11}{}%
    \setoutput{12}{}%
    \setoutput{13}{}%
    \setoutput{14}{}%
    \setoutput{15}{}%
    \setoutput{16}{}%
    \setoutput{17}{}%
    \setoutput{18}{}%
    \setoutput{19}{}%
    \setoutput{20}{}%
    \setoutput{21}{}%
    \setoutput{22}{}%
    \setoutput{23}{}%
    \setoutput{24}{}%
    \setoutput{25}{}%
    \setoutput{26}{}%
    \setoutput{27}{}%
    \setoutput{28}{}%
    \setoutput{29}{}%
    \setoutput{30}{}%
    \setoutput{31}{}%
  }%
  % --------
  % function
  % --------
  %
  % #1  name of the function
  %
  \newcommand{\function}[1]
  {%
    \renewcommand{\valuefname}{##1}%
  }%
  % ------
  % inputA
  % ------
  %
  % set the name of the input variable that toggles with the lowest frequency
  %
  % #1  name of the variable
  %
  \newcommand{\inputA}[1]
  {%
    \renewcommand{\valueinputA}{##1}%
  }%
  % ------
  % inputB
  % ------
  %
  % #1  name of the variable
  %
  \newcommand{\inputB}[1]
  {%
    \renewcommand{\valueinputB}{##1}%
  }%
  % ------
  % inputC
  % ------
  %
  % #1  name of the variable
  %
  \newcommand{\inputC}[1]
  {%
    \renewcommand{\valueinputC}{##1}%
  }%
  % ------
  % inputD
  % ------
  %
  % #1  name of the variable
  %
  \newcommand{\inputD}[1]
  {%
    \renewcommand{\valueinputD}{##1}%
  }%
  % ------
  % inputE
  % ------
  %
  % set the name of the input variable that toggles with the highest frequency
  %
  % #1  name of the variable
  %
  \newcommand{\inputE}[1]
  {%
    \renewcommand{\valueinputE}{##1}%
  }%
  % -------
  % minterm
  % -------
  %
  % #1  minterm number 0 ... 31
  %
  \newcommand{\minterm}[1]
  {%
    \setoutput{##1}{1}%
  }%
  % -------
  % maxterm
  % -------
  %
  % #1  maxterm number 0 ... 31
  %
  \newcommand{\maxterm}[1]
  {%
    \setoutput{##1}{0}%
  }%
  % --------
  % dontcare
  % --------
  %
  % #1  don't care number 0 ... 31
  %
  \newcommand{\dontcare}[1]
  {%
    \setoutput{##1}{X}%
  }%
  % -----
  % areas
  % -----
  %
  % define areas to fill
  %
  \newcommand{\areas}[1]
  {%
    \renewcommand{\printareas}{##1}
  }%
  % --------
  % colorize
  % --------
  %
  % #1 top row          (1..m)
  % #2 left column      (1..n)
  % #3 width            (number of columns)
  % #4 height           (number of rows)
  % #5 top    reduction (pt, cm, ...)
  % #6 bottom reduction (pt, cm, ...)
  % #7 left   reduction (pt, cm, ...)
  % #8 right  reduction (pt, cm, ...)
  % #9 color
  %
  \newcommand{\colorize}[9]
  {%
    \begin{scope}[xshift=0cm + (##2 - 1) * 1cm, yshift=4cm - (##1 - 1) * 1cm]
      % background
      \ifthenelse{\equal{\showtype}{FILL}}%
      {%
        \fill
        [
          opacity=0.5,
          fill=##9,
          rounded corners
        ] (0cm + ##7, 0cm - ##5) rectangle (1cm * ##3 - ##8, -1cm * ##4 + ##6);
      }%
      {\relax}%
      % border
      \ifthenelse{\equal{\showtype}{DRAW}}%
      {%
        \draw
        [
          draw=##9,
          line width=1pt,
          rounded corners
        ] (0cm + ##7, 0cm - ##5) rectangle (1cm * ##3 - ##8, -1cm * ##4 + ##6);
      }%
      {\relax}%
    \end{scope}
  }%
  % ----------------------------------------------------------------------------
  % Internal functions                                        Internal functions
  % ----------------------------------------------------------------------------
  % ---------
  % setoutput
  % ---------
  %
  % used my minterm, maxterm and dontcare
  %
  % #1  index: 0 .. 31
  % #2  text
  %
  \newcommand{\setoutput}[2]
  {%
    % map index to matrix position
    \ifthenelse{\equal{##1}{0}} {\renewcommand{\valueoutputAA}{##2}}{\relax}%
    \ifthenelse{\equal{##1}{1}} {\renewcommand{\valueoutputAB}{##2}}{\relax}%
    \ifthenelse{\equal{##1}{2}} {\renewcommand{\valueoutputAD}{##2}}{\relax}%
    \ifthenelse{\equal{##1}{3}} {\renewcommand{\valueoutputAC}{##2}}{\relax}%
    \ifthenelse{\equal{##1}{4}} {\renewcommand{\valueoutputBA}{##2}}{\relax}%
    \ifthenelse{\equal{##1}{5}} {\renewcommand{\valueoutputBB}{##2}}{\relax}%
    \ifthenelse{\equal{##1}{6}} {\renewcommand{\valueoutputBD}{##2}}{\relax}%
    \ifthenelse{\equal{##1}{7}} {\renewcommand{\valueoutputBC}{##2}}{\relax}%
    \ifthenelse{\equal{##1}{8}} {\renewcommand{\valueoutputDA}{##2}}{\relax}%
    \ifthenelse{\equal{##1}{9}} {\renewcommand{\valueoutputDB}{##2}}{\relax}%
    \ifthenelse{\equal{##1}{10}}{\renewcommand{\valueoutputDD}{##2}}{\relax}%
    \ifthenelse{\equal{##1}{11}}{\renewcommand{\valueoutputDC}{##2}}{\relax}%
    \ifthenelse{\equal{##1}{12}}{\renewcommand{\valueoutputCA}{##2}}{\relax}%
    \ifthenelse{\equal{##1}{13}}{\renewcommand{\valueoutputCB}{##2}}{\relax}%
    \ifthenelse{\equal{##1}{14}}{\renewcommand{\valueoutputCD}{##2}}{\relax}%
    \ifthenelse{\equal{##1}{15}}{\renewcommand{\valueoutputCC}{##2}}{\relax}%
    \ifthenelse{\equal{##1}{16}}{\renewcommand{\valueoutputAH}{##2}}{\relax}%
    \ifthenelse{\equal{##1}{17}}{\renewcommand{\valueoutputAG}{##2}}{\relax}%
    \ifthenelse{\equal{##1}{18}}{\renewcommand{\valueoutputAE}{##2}}{\relax}%
    \ifthenelse{\equal{##1}{19}}{\renewcommand{\valueoutputAF}{##2}}{\relax}%
    \ifthenelse{\equal{##1}{20}}{\renewcommand{\valueoutputBH}{##2}}{\relax}%
    \ifthenelse{\equal{##1}{21}}{\renewcommand{\valueoutputBG}{##2}}{\relax}%
    \ifthenelse{\equal{##1}{22}}{\renewcommand{\valueoutputBE}{##2}}{\relax}%
    \ifthenelse{\equal{##1}{23}}{\renewcommand{\valueoutputBF}{##2}}{\relax}%
    \ifthenelse{\equal{##1}{24}}{\renewcommand{\valueoutputDH}{##2}}{\relax}%
    \ifthenelse{\equal{##1}{25}}{\renewcommand{\valueoutputDG}{##2}}{\relax}%
    \ifthenelse{\equal{##1}{26}}{\renewcommand{\valueoutputDE}{##2}}{\relax}%
    \ifthenelse{\equal{##1}{27}}{\renewcommand{\valueoutputDF}{##2}}{\relax}%
    \ifthenelse{\equal{##1}{28}}{\renewcommand{\valueoutputCH}{##2}}{\relax}%
    \ifthenelse{\equal{##1}{29}}{\renewcommand{\valueoutputCG}{##2}}{\relax}%
    \ifthenelse{\equal{##1}{30}}{\renewcommand{\valueoutputCE}{##2}}{\relax}%
    \ifthenelse{\equal{##1}{31}}{\renewcommand{\valueoutputCF}{##2}}{\relax}%
  }%
  % -----------
  % printlabels
  % -----------
  %
  % print all input labels
  %
  \newcommand{\printlabels}
  {%
    \draw[|-|] ( 8.3,  0.0) -- node[right] {\valueinputB} ( 8.3,  2.0);
    \draw[|-|] (-0.3,  1.0) -- node[left]  {\valueinputC} (-0.3,  3.0);
    \draw[|-|] ( 1.0, -0.3) -- node[below] {\valueinputE} ( 3.0, -0.3);
    \draw[|-|] ( 5.0, -0.3) -- node[below] {\valueinputE} ( 7.0, -0.3);
    \draw[|-|] ( 2.0,  4.3) -- node[above] {\valueinputD} ( 6.0,  4.3);
    \draw[|-|] ( 4.0,  5.0) -- node[above] {\valueinputA} ( 8.0,  5.0);
  }%
  % ------------
  % printindices
  % ------------
  %
  % print all index numbers
  %
  \newcommand{\printindices}
  {%
    \begin{scope}[xshift=2mm, yshift=8mm]
      % 1. row (top)
      \node at (0, 3) {{\tiny\sffamily  0}};
      \node at (1, 3) {{\tiny\sffamily  1}};
      \node at (2, 3) {{\tiny\sffamily  3}};
      \node at (3, 3) {{\tiny\sffamily  2}};
      \node at (4, 3) {{\tiny\sffamily 18}};
      \node at (5, 3) {{\tiny\sffamily 19}};
      \node at (6, 3) {{\tiny\sffamily 17}};
      \node at (7, 3) {{\tiny\sffamily 16}};
      % 2. row
      \node at (0, 2) {{\tiny\sffamily  4}};
      \node at (1, 2) {{\tiny\sffamily  5}};
      \node at (2, 2) {{\tiny\sffamily  7}};
      \node at (3, 2) {{\tiny\sffamily  6}};
      \node at (4, 2) {{\tiny\sffamily 22}};
      \node at (5, 2) {{\tiny\sffamily 23}};
      \node at (6, 2) {{\tiny\sffamily 21}};
      \node at (7, 2) {{\tiny\sffamily 20}};
      % 3. row
      \node at (0, 1) {{\tiny\sffamily 12}};
      \node at (1, 1) {{\tiny\sffamily 13}};
      \node at (2, 1) {{\tiny\sffamily 15}};
      \node at (3, 1) {{\tiny\sffamily 14}};
      \node at (4, 1) {{\tiny\sffamily 30}};
      \node at (5, 1) {{\tiny\sffamily 31}};
      \node at (6, 1) {{\tiny\sffamily 29}};
      \node at (7, 1) {{\tiny\sffamily 28}};
      % 4. row (bottom)
      \node at (0, 0) {{\tiny\sffamily  8}};
      \node at (1, 0) {{\tiny\sffamily  9}};
      \node at (2, 0) {{\tiny\sffamily 11}};
      \node at (3, 0) {{\tiny\sffamily 10}};
      \node at (4, 0) {{\tiny\sffamily 26}};
      \node at (5, 0) {{\tiny\sffamily 27}};
      \node at (6, 0) {{\tiny\sffamily 25}};
      \node at (7, 0) {{\tiny\sffamily 24}};
    \end{scope}
  }%
  % -----------
  % printoutput
  % -----------
  %
  % show all output values
  %
  \newcommand{\printoutput}
  {%
    \begin{scope}[xshift=0.5cm, yshift=0.5cm]
      % 1. row (top)
      \node at (0, 3) {\valueoutputAA};
      \node at (1, 3) {\valueoutputAB};
      \node at (2, 3) {\valueoutputAC};
      \node at (3, 3) {\valueoutputAD};
      \node at (4, 3) {\valueoutputAE};
      \node at (5, 3) {\valueoutputAF};
      \node at (6, 3) {\valueoutputAG};
      \node at (7, 3) {\valueoutputAH};
      % 2. row
      \node at (0, 2) {\valueoutputBA};
      \node at (1, 2) {\valueoutputBB};
      \node at (2, 2) {\valueoutputBC};
      \node at (3, 2) {\valueoutputBD};
      \node at (4, 2) {\valueoutputBE};
      \node at (5, 2) {\valueoutputBF};
      \node at (6, 2) {\valueoutputBG};
      \node at (7, 2) {\valueoutputBH};
      % 3. row
      \node at (0, 1) {\valueoutputCA};
      \node at (1, 1) {\valueoutputCB};
      \node at (2, 1) {\valueoutputCC};
      \node at (3, 1) {\valueoutputCD};
      \node at (4, 1) {\valueoutputCE};
      \node at (5, 1) {\valueoutputCF};
      \node at (6, 1) {\valueoutputCG};
      \node at (7, 1) {\valueoutputCH};
      % 4. row (bottom)
      \node at (0, 0) {\valueoutputDA};
      \node at (1, 0) {\valueoutputDB};
      \node at (2, 0) {\valueoutputDC};
      \node at (3, 0) {\valueoutputDD};
      \node at (4, 0) {\valueoutputDE};
      \node at (5, 0) {\valueoutputDF};
      \node at (6, 0) {\valueoutputDG};
      \node at (7, 0) {\valueoutputDH};
    \end{scope}
  }%
  % ----------
  % printareas
  % ----------
  %
  % show all colorized areas
  %
  \newcommand{\printareas}{\relax}%
  % ----------------------------------------------------------------------------
  % Attributes                                                        Attributes
  % ----------------------------------------------------------------------------
  % some common colors
  \definecolor{C1}{named}{Red}%
  \definecolor{C2}{named}{Green}%
  \definecolor{C3}{named}{Blue}%
  \definecolor{C4}{named}{Sepia}%
  \definecolor{C5}{named}{Cyan}%
  \definecolor{C6}{named}{Magenta}%
  \definecolor{C7}{named}{Yellow}%
  \definecolor{C8}{named}{Gray}%
  % flags
  \newcommand{\showlabels}{TRUE}%
  \newcommand{\showindices}{TRUE}%
  % name of the function
  \newcommand{\valuefname}{\relax}%
  % names of the five input variables
  \newcommand{\valueinputA}{$e_4$}%
  \newcommand{\valueinputB}{$e_3$}%
  \newcommand{\valueinputC}{$e_2$}%
  \newcommand{\valueinputD}{$e_1$}%
  \newcommand{\valueinputE}{$e_0$}%
  % output values (matrix positions)
  \newcommand{\valueoutputAA}{$0$}% 0
  \newcommand{\valueoutputAB}{$0$}% 1
  \newcommand{\valueoutputAC}{$0$}% 3
  \newcommand{\valueoutputAD}{$0$}% 2
  \newcommand{\valueoutputAE}{$0$}% 18
  \newcommand{\valueoutputAF}{$0$}% 19
  \newcommand{\valueoutputAG}{$0$}% 17
  \newcommand{\valueoutputAH}{$0$}% 16
  \newcommand{\valueoutputBA}{$0$}% 4
  \newcommand{\valueoutputBB}{$0$}% 5
  \newcommand{\valueoutputBC}{$0$}% 7
  \newcommand{\valueoutputBD}{$0$}% 6
  \newcommand{\valueoutputBE}{$0$}% 22
  \newcommand{\valueoutputBF}{$0$}% 23
  \newcommand{\valueoutputBG}{$0$}% 21
  \newcommand{\valueoutputBH}{$0$}% 20
  \newcommand{\valueoutputCA}{$0$}% 12
  \newcommand{\valueoutputCB}{$0$}% 13
  \newcommand{\valueoutputCC}{$0$}% 15
  \newcommand{\valueoutputCD}{$0$}% 14
  \newcommand{\valueoutputCE}{$0$}% 30
  \newcommand{\valueoutputCF}{$0$}% 31
  \newcommand{\valueoutputCG}{$0$}% 29
  \newcommand{\valueoutputCH}{$0$}% 28
  \newcommand{\valueoutputDA}{$0$}% 8
  \newcommand{\valueoutputDB}{$0$}% 9
  \newcommand{\valueoutputDC}{$0$}% 11
  \newcommand{\valueoutputDD}{$0$}% 10
  \newcommand{\valueoutputDE}{$0$}% 26
  \newcommand{\valueoutputDF}{$0$}% 27
  \newcommand{\valueoutputDG}{$0$}% 25
  \newcommand{\valueoutputDH}{$0$}% 24
}%
{%
  \begingroup
    % show colored areas first (lowest layer)
    \begin{scope}
      % clip colored areas
      \clip (-0.15, -0.15) rectangle (8.15, 4.15);
      % show background
      {\newcommand{\showtype}{FILL}\printareas}%
      % show border
      {\newcommand{\showtype}{DRAW}\printareas}%
    \end{scope}
    % show grid
    \draw (0, 0) grid (8, 4);
    % name of the function
    \ifthenelse{\equal{\valuefname}{\relax}}{\relax}{\node at (-0.5, 4.5) {\valuefname};}
    % labels
    \ifthenelse{\equal{\showlabels}{TRUE}}{\printlabels}{\relax}
    % indices
    \ifthenelse{\equal{\showindices}{TRUE}}{\printindices}{\relax}
    % output values
    \printoutput
  \endgroup
}%
